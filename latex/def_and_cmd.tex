\newtheorem{remark}{\textbf{Remark}}

% symbol
\def \mL{{\mathcal L}}
\def \mS{{\mathcal S}}
\def \z{{\bf z}}
\def \v{{\bf v}}
\def \c{{\bf c}}
\def \t{{\bf t}}
\def \x{{\bf x}}
\def \y{{\bf y}}
\def \p{{\bf p}}
\def \V{{\bf V}}
% \def \hV{\hat{{\bf V}}}
\def \tV{\tilde{{\bf V}}}
\def \bv{\bar{{\bf v}}}
\def \bt{\bar{{\bf t}}}
\def \mR{\mathbb{R}}

\usepackage{pifont}
\definecolor{my_green}{RGB}{51,102,0}
\definecolor{my_red}{RGB}{204, 0, 0}
\newcommand{\cmark}{\textcolor{my_green}{\ding{51}}} % ✔
\newcommand{\xmark}{\textcolor{my_red}{\ding{55}}} % ✘

% style
\newcommand{\acronym}[1]{\underline{\textbf{#1}}}
\newcommand{\m}[1]{\mathbf{#1}}
\newcommand{\rotbox}[1]{\rotatebox{90}{#1}}

% variable
\newcommand{\modelname}{\textbf{PASTA}\xspace}

% color
\newcommand{\demph}[1]{\textcolor{demphcolor}{#1}}
\newcommand{\impro}[1]{{\hspace{0.05cm}{\color[HTML]{32CB00}\textbf{(+#1)}}}}
\newcommand{\degra}[1]{{\hspace{0.05cm}{\textcolor{red}{\textbf{(-#1)}}}}}
\rowcolor{gray!25}  % set background color of a row in a table to gray. require \usepackage{colortbl}

% colorbox
\definecolor{oorange}{RGB}{252,218,227}
\definecolor{yyellow}{RGB}{255,237,203}
\definecolor{ppurple}{RGB}{208,205,226}
\definecolor{ggreen}{RGB}{195,222,176}
% example usage: \colorbox{ggreen}{todo}

% icons
\newcommand{\snowflake}{\includegraphics[width=10px]{icons/snowflake.png}}
\newcommand{\fire}{\includegraphics[width=10px]{icons/fire.jpg}}

% comment
\newcommand{\rsh}[1]{\textcolor{orange}{[#1 - rsh]}}

% change caption step in supplementary materials or rebuttal
\captionsetup[figure]{name={Sup.Fig.}}
\captionsetup[table]{name={Sup.Tab.}}

% Two parallel Figures
\begin{figure*}
\centering
\begin{minipage}[b]{0.5\textwidth}
    \includegraphics[width=\textwidth]{icons/snowflake.png}
    \caption{}
    \label{fig:1}
\end{minipage}
\hspace{5pt} % Add 1cm of space
\begin{minipage}[b]{0.5\textwidth}
    \includegraphics[width=\textwidth]{icons/fire.jpg}
    \caption{}
    \label{fig:2}
\end{minipage}
\end{figure*}

% Two parallel sub-figures (a) (b) in one figure
\begin{figure*}[tbp]
  \centering
  \subfigure[sfig-cap-1]{
    \label{sfig:1}
    \includegraphics[width=.48\textwidth]{icons/snowflake.png}} %\hspace{1em}
  \subfigure[CIFAR100]{
    \label{sfig-cap-2}
    \includegraphics[width=.45\textwidth]{icons/fire.jpg}}
  \caption{}
  \label{fig:}
\end{figure*}

% wrapfigure
\begin{wrapfigure}[19]{tr}{0.5\textwidth}
\centering
\includegraphics[width=0.5\textwidth]{icons/snowflake.png}
\caption{}
\label{fig:overview}
\end{wrapfigure}

% package
\usepackage{hyperref}
\usepackage{graphics}
\usepackage[utf8]{inputenc} % allow utf-8 input
\usepackage[T1]{fontenc}    % use 8-bit T1 fonts
\usepackage{hyperref}       % hyperlinks
\usepackage{url}            % simple URL typesetting
\usepackage{booktabs}       % professional-quality tables
\usepackage{amsfonts}       % blackboard math symbols
\usepackage{nicefrac}       % compact symbols for 1/2, etc.
\usepackage{microtype}      % microtypography
\usepackage{xcolor}         % colors
% \usepackage[square,sort,comma,numbers]{natbib}
\usepackage[utf8]{inputenc}
\usepackage{graphicx}
\usepackage{amsmath}
\usepackage{amssymb}
\usepackage{mathtools}
\usepackage{amsthm}
\usepackage{arydshln}
\usepackage{multirow}
\usepackage{wrapfig, lipsum, booktabs}
\usepackage{algorithm}
\usepackage{enumitem}
\usepackage{paralist, tabularx}
\usepackage{balance}
%\usepackage{algorithmic}
\usepackage[noend]{algpseudocode}
\usepackage{pgfplots}
\usetikzlibrary{pgfplots.groupplots}
\pgfplotsset{compat=1.3}
\usepackage{tikz}
\usetikzlibrary{patterns}
\usepackage{pgf-pie}
\usepackage{adjustbox}
\usepackage{colortbl}
